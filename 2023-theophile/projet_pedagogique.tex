% !TeX TS-program = lualatex
\documentclass[french]{article}
% \usepackage{fontspec}
\usepackage[a4paper]{geometry}
\usepackage[frenchb]{babel}
\usepackage{todonotes}
\usepackage{easyReview}

\title{Dossier Pédagogique}

\author{Théophile Cram}

\date{Année scolaire 2023-2024}

\begin{document}
	
	\maketitle
	
	
	\section{Choix pédagogiques}
	
	\subsection{Démarche}
	Le site www.service-public.fr précise que \og{}\textit{les parents doivent veiller à l'éducation intellectuelle, professionnelle, civique [de l'enfant]}\fg{}. En tant que parents de Théophile, nous avons effectivement envers lui \remove{un devoir de protection matériel et moral, un devoir de gestion du patrimoine et }un devoir d'éducation.

	Devant l'importance de l'enjeu, nous avons décidé d'assumer entièrement cette \replace{éducation}{mission} et de ne pas la déléguer à un établissement scolaire, afin de garder le contrôle sur ce qui sera soumis à sa sensibilité et à son intelligence. \\

	\subsection{L'environnement de Théophile}
	
	\add{Depuis 2018}, nous avons fait le choix \replace{familial}{parental} \remove{en 2018} de nous occuper nous-mêmes leur instruction. Pour ce faire, mon mari a décidé \add{depuis cette année-là} d'assurer seul les revenus du foyer \replace{afin que je puisse arrêter mon activité professionnelle et consacrer mon temps à nos enfants}{tandis que j'arrêtais toute activité professionnelle pour me consacrer à temps plein à cette mission}.\\
	
	\replace{Nous bénéficions pour les années scolaires 2022/2023 et 2023/2024 d'une dérogation permettant à Valentin (10 ans), Agathe (7 ans) et Constance (5 ans) de poursuivre leur instruction à la maison.}{Théophile est le cinquième d'une fratrie de six enfants. Durant sa journée d'instruction en famille, Théophile évolue et travaille avec Valentin (10 ans), Agathe (7 ans) et Constance (5 ans), recevant eux-même l'instruction en famille.} 
	
	La dynamique fraternelle est centrale dans notre pédagogie concernant le premier cycle. Le jeu avec ses frères et soeurs, qu'il soit libre, orienté ou de société, est pour Théophile source de découvertes et d'apprentissages.
	
	\section{Ressources et supports éducatifs utilisés}
	\subsection{Mobiliser le langage dans toutes ses dimensions}
	\subsubsection{Langage oral}
	Le développement du langage passe d'abord par les jeux d'imagination qu'il partage avec la fratrie. Citons notamment :
	\begin{itemize}
		\item les jeux de Playmobil,
		\item les déguisements,
		\item les jeux de Légos,
		\item les voitures,
		\item les dînettes,
		\item etc.
	\end{itemize}
	Au cours de ses jeux, Théophile invente et verbalise des histoires cohérentes qu'il \replace{m'expose}{expose à moi-même,} ainsi qu'à ses frères et soeurs. Il est alors amené à argumenter, négocier ou consolider son récit avec le reste de la fratrie.
		\subsubsection{Lecture}
	La lecture quotidienne des histoires par les parents ou la fratrie est également l'occasion pour Théophile de s'exprimer à l'oral, poser des questions et commenter l'histoire, \replace{pour progressivement la raconter lui-même}{puis éventuellement, progressivement, la restituer et la raconter lui-même}. Nous disposons pour cela de nombreux ouvrages de littérature enfantine\replace{, et Théophile}{. Théophile} est \add{d'ailleurs} abonné à un magazine de son âge.
			\subsubsection{Ecriture}
	La reconnaissance de l'alphabet, en \replace{capital}{lettres capitales} dans un premier temps, puis en script et en cursive \add{dans un deuxième temps} passe par des jeux de mémorisation et de reconnaissance.\\
	
	Pour le langage écrit, la motricité fine et le graphisme seront travaillés sur des supports variés :
	\begin{itemize}
		\item semoule fine sur laquelle il peut laisser les traces de ses doigts,
		\item peinture,
		\item coloriage,
		\item perles à enfiler (d'abord de grande taille, puis de plus en plus fines),
		\item perles à repasser,
		\item collage, 
		\item découpage,
		\item pâte à modeler\ldots{}
	\end{itemize}

Pour le moment, Théophile tient difficilement les crayons et commence juste à découper des bandes de papier. En fonction de sa maîtrise, nous pourrons faire évoluer les supports, permettant à Théophile de contrôler son geste de plus en plus finement. Des fiches d'activité glanées sur internet (les Coccinelles, la Boîte à Bons Points) viendront consolider les acquis en marquant de manière formelle les étapes franchies.
	
	\subsection{Agir, s'exprimer et comprendre}
	\subsubsection{L'activité physique}
	Les déguisements sont pour Théophile l'occasion d'explorer ses capacités physiques : imitation de la grenouille, du loup, du cheval, du chat\ldots{}\\
	
	Les sorties quotidiennes au parc lui permettent de courir, de sauter et de se déplacer dans des cadres différents. Il dispose pour cela d'une trottinette et d'un vélo.\\
	
	Nous prévoyons également de l'inscrire à un cours de natation pour le familiariser avec l'environnement aquatique.
		
	\subsubsection{Les activités artistiques}
	En plus des différents supports artistiques mobilisés pour le travail de la motricité fine, nous attachons une grande importance à la découverte de la musique. Il est amené à écouter et chanter des comptines, chansons et chants religieux quotidiennement, avec la présence d'un lecteur CD dans sa chambre.
	
	Nous sommes attentifs au développement du sens du Beau chez les enfants, et nous observerons les talents que Théophile pourra développer pour exprimer au mieux sa créativité : soit par la musique ou la danse avec le Conservatoire, soit par le modelage, le dessin ou autres activité artistiques proposées par des associations nantaises. 
	
	Enfin, il participera à la Crèche Vivante pour la veillée de Noël.\\


	
	
	\subsection{Acquérir les premiers outils mathématiques}
	
	\subsubsection{Les nombres}
	Les différents jeux de société sont toujours très utiles pour consolider la conscience des nombres, la comparaison et la décomposition.\\
	
	Théophile connaît déjà la comptine numérique jusqu'à 15 et sait dénombrer jusqu'à 8. Cet apprentissage est tiré directement des jeux qu'il partage avec des soeurs plus âgées (cache-cache, petits chevaux, jeux de cartes, mémo, etc). Il est régulièrement amené à devoir dénombrer les cartes ou les pions qu'il a remportés, et qu'il doit comparer avec les \og{}butins\fg{} de ses soeurs.\\
	
	
		\subsubsection{Formes, grandeurs, suites organisées}
La reconnaissance des formes et des algorithmes passera elle aussi essentiellement par le jeu :
 \begin{itemize}
 	\item puzzles,
 	\item cubes,
 	\item kaplas,
 	\item jeux de construction,
 	\item petit train à monter,
	\item etc
 \end{itemize}
 
 \subsection{Explorer le monde}
 
 \subsubsection{Le temps}
 
 Nous aidons Théophile à se repérer dans le temps par des rituels quotidiens autour :
 \begin{itemize}
 	\item de la date : il possède un petit calendrier perpétuel que nous utiliserons quotidiennement 
 	\item des jours de la semaine : jours de travail, jours d'activités, jours de repos,
 	\item des moments de la journée, par la distinction des différents repas, de ce qui vient avant et après.
 \end{itemize}
Il arrive à situer le passé et le présent, et commence à intégrer la notion de journée. \\
 
 \subsubsection{Le potager}
 En outre, nous sommes membres de l'association Papotager, grâce à laquelle nous disposons d'une parcelle de 5 m$^{2}$ à cultiver derrière la médiathèque Jacques Demy. C'est pour Théophile une source intarissable d'apprentissages :
 \begin{itemize}
 	\item rythme des saisons,
 	\item observation de l'évolution des plantes depuis le semis jusqu'au fruit (semis qu'il aura lui-même entretenu puis mis en terre),
 	\item observation des insectes et autres animaux qui peuplent le jardin, et réflexion sur leur rôle ou leur nuisance dans le développement du potager. 	
 \end{itemize}


 \subsubsection{La vie quotidienne}
Théophile prend part avec plaisir à chacune des tâches de la vie quotidienne, ce qui est toujours riche en apprentissages et en développements.\\

Par exemple, comme nous avons la chance d'habiter un quartier abondant en petits commerces (boucherie, crêmerie, poissonnerie, boulangerie, pâtisserie\ldots{}), Théophile est toujours ravi de m'accompagner faire les courses nécessaires pour la préparation du déjeuner, du goûter ou du dîner. Au retour à la maison, il m'observe et participe à l'élaboration du repas. De cette manière, il comprend d'où vient sa nourriture (quels animaux, quelles plantes\ldots{}) et  comment elle est préparée. Plus tard, il sera également capable d'équilibrer un repas et de proposer un menu.\\

\subsection{Vivre ensemble}
		Théophile vit selon les principes du cathéchisme catholique. Nous cherchons à développer chez lui les vertus de justice et de charité envers le prochain. Nous lui apprenons également en quoi la vie est respectable, et lui expliquons pourquoi \add{des actes comme} la cruauté envers les animaux, le gaspillage et la destruction gratuite des plantes \replace{est interdite}{sont mauvais et donc interdits}.\\
		
			En outre, son quotidien au sein d'une famille nombreuse lui apprend à écouter, à partager, à s'exprimer devant un petit groupe, ainsi qu'à patienter. Cette vie de famille l'encourage également à l'autonomie, par les tâches qu'il a à accomplir quotidiennement pour participer à l'effort collectif (ranger ses légos et son train, aider à débarrasser la table\ldots{})\\
		
		Théophile est un petit garçon impatient de devenir autonome, et toujours enthousiaste à rendre service lorsqu'un frère ou une soeur le lui demande. 
	
	\section{Organisation du temps de l'enfant}

\subsection{Organisation de la journée}


\begin{tabular}{|r|c|}
	\hline
	9h00 - 10h30 & Jeux - activités libres \\
	\hline
	10h30 - 11h30 & Activités encadrées non formelles (cuisine, courses, expériences\ldots{}) \\
	\hline
	11h30 - 13h00 & Déjeuner \\
	\hline
	13h00 - 14h00 & Sieste ou temps calme \\
	\hline
	14h00 - 15h30 & Activités encadrées formelles ou sieste  \\
	\hline
	15h30 - 16h00 & Goûter \\
	\hline
	16h00 - 18h00 & Sortie au parc \\
	\hline
\end{tabular}


\subsection{Organisation de l'année}

Théophile suivra un cours de natation hebdomadaire à partir de la rentrée.

Nous fréquentons régulièrement les musées de Nantes, Théophile visitera au moins une fois dans l'année :
\begin{itemize}
	\item le Muséum d'Histoire Naturelle,
	\item le Château des Ducs de Bretagne.
\end{itemize}

Nous sommes attentifs aux ateliers proposés aux enfants pendant les vacances scolaires par les musées ou les associations de l'agglomération (Musée des Beaux-Arts, Musée de l'Imprimerie, l'atelier d'argile Trois8, l'Abord'âge, etc.)

Nous prévoyons aussi de participer à une ou deux sorties pédagogiques organisées par l'association IEF 44 (les années précédentes, nous avons pu fabriquer un Cerf-Volant à l'Atelier du Vent avec les aînés de la fratrie, visiter l'Historial de Vendée, participer à une sortie en bâteau à Oudon sur le thème de la faune ligurienne\ldots{}).


\end{document}
